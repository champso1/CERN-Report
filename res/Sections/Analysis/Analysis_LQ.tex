\subsection{LQ Analysis}
    Unfortunately, the requested LQ samples were not completed in time for an analysis to be run on them. The formalities associated with gaining approval from the various convening groups as well as the requesting process took up far more time than was initially expected. Further, I had to learn all about the theory behind LQ's and how to operate the Athena software. Despite this, it was still very much worthwhile to spend the time porting the Rel.21 $2\ell SS + 1\tau$ analysis code into Rel.22 and testing it on the existing $t\bar{t}H$ samples. We found that even with a simple neural network, we were able to get some relatively good results, which bode well both for the quality of Rel.22 samples but also for the future analysis of the LQ samples.