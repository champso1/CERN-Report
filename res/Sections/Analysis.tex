\section{Analysis}
    There are a number of SM processes that contribute also to the $2\ell SS + 1\tau$ channel, including Higgs production, heavy boson ($W^{\pm},\ Z$) production, top quark production, and several others; the full table of background (and signal) processes that are considered for the analysis is given in Table~\ref{signalAndBackgroundDSIDs} in Appendix~\ref{DSIDs}. Because of this, we need to find a way to separate these ordinary SM ``backgrounds'' from our LQ pair production ``signal'', and the best way to achieve this is with machine learning. This is because there are dozens of relevent features that are present in the ntuples created from the event generation described in Section~\ref{methodology}, and many machine learning models have an edge over conventional fitting algorithms when it comes to many-dimensional scenarios such as this.

    This process is done in a number of different ways based on the experiment -- we will briefly list our steps here then fully describe them in the subsequent sections. First, we make some preliminary plots/tables using TRExFitter to show the raw number of events with and without weighting/selection criteria. Then, we produce so-called ``small'' ntuples which use the same selection criteria to slim down the raw ntuples and make further analysis quicker. At the same time, we take the full dataset and transform it to numpy arrays so that it can be used to train a machine learning model. After the model is trained, the small ntuples are fed through it to generate probabilities in what are called ``friend'' ntuples. Lastly, TRExFitter can generate full distributions and other statistical measures using the friend-ntuples.
    

    \import{./res/Sections/Analysis}{Analysis_ttH.tex}
    \import{./res/Sections/Analysis}{Analysis_LQ.tex}