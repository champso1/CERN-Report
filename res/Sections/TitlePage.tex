\begin{titlepage}
    \begin{center}
        \vspace*{8mm}

        \includegraphics[width=0.2\linewidth]{./res/TitlePagePics/CERNLogo.jpg}
        \includegraphics[width=0.2\linewidth]{./res/TitlePagePics/ATLASLogo.png}

        \vspace*{8mm}
            
        \Huge
        \textbf{Analysis Sensitivity Test in the $2\ell SS + 1\tau$ Channel for Asymmetric Leptoquark Production Mechanism}
            
        \vspace{1.8cm}
         
        \LARGE
        \textbf{Casey Hampson} \\
        \Large
        \textit{Department of Physics}\\
        \textit{Kennesaw State University, USA}\\
        \vspace*{5mm}
        \textbf{Andr\'e Sopczak} \\
        \Large
        \textit{University of Prague, Czechia}\\
        

        \vfill

        \begin{center}
            \Large
            \textbf{Abstract} \\
            \large
            In this report, we briefly describe the phenomenology behind a novel method for pair producing asymmetric leptoquarks (LQ's), in which the final state LQ's are not charge conjugates of each other. Then, we outline the process of testing such a model from within ATLAS's Athena software, generating validation plots, and submitting offical Monte Carlo (MC) samples from the computing grid. From there, we describe the analysis done for the previous release (Rel.) 21 ntuples in the $2\ell SS + 1\tau$ channel with $t\bar{t}H$ production as signal, and how it was ported to the current Rel.22 ntuples. A case study was done with the $t\bar{t}H$ samples to test the new pipeline. All that remains is to input the LQ signal once it is produced.
        \end{center}
            
        \large
        \rule{0.7\linewidth}{0.5pt} \\
        CERN Summer Student Project Report \\
        \textit{August 09, 2024}
        \vspace{0.2cm}
        
            
    \end{center}
\end{titlepage}