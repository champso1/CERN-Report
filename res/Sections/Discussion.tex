\section{Discussion}
    For this summer student project, we investigated a novel method of leptoquark pair production which involves two leptoquarks in the final state that are not charge conjugates of each other. Additionally, there is the added benefit of a quark-quark initial state, which is desirable at the LHC. We outlined the steps taken to test the model and submit an official MC request on the grid for larger samples. A $t\bar{t}H$ analysis from Rel.21 ntuples was ported to Rel.22, which involved changing/removing a large number of branches/features from the pre-selection and training processes. This analysis was then done for the existing $t\bar{t}H$ Rel.22 samples, and relatively good agreement was found even for a simple neural network.
    
    The next steps for the project would be to generate the full set of LQ samples for mass points in the range 1500-2500 GeV and for Yukawa couplings of 0.5 and 1.0. Then, the current $t\bar{t}H$ signal samples would be moved to background, and the new LQ samples become the new signal. The whole pipeline should work without any additional changes, so further hyperparameter tuning in terms of the machine learning model and selection criteria can be done. For instance, as mentioned before, PLV variables, which are helpful in removing fakes, can be added once they (or a viable substitute) are developed for Rel.22. This will assist in reducing backgrouns like $t\bar{t}$ which are seen to be quite relevent in the final plots.